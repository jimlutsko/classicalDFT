\documentclass[aps,preprint]{revtex4-1}%
\usepackage{amssymb}
\usepackage{amsfonts}
\usepackage{amsmath}
\usepackage{graphicx}%
\setcounter{MaxMatrixCols}{30}
%TCIDATA{OutputFilter=latex2.dll}
%TCIDATA{Version=5.50.0.2960}
%TCIDATA{CSTFile=revtex4-1.cst}
%TCIDATA{Created=Thursday, August 19, 2021 13:56:05}
%TCIDATA{LastRevised=Thursday, August 19, 2021 14:34:02}
%TCIDATA{<META NAME="GraphicsSave" CONTENT="32">}
%TCIDATA{<META NAME="SaveForMode" CONTENT="1">}
%TCIDATA{BibliographyScheme=Manual}
%TCIDATA{<META NAME="DocumentShell" CONTENT="Articles\SW\REVTeX 4-1">}
%BeginMSIPreambleData
\providecommand{\U}[1]{\protect\rule{.1in}{.1in}}
%EndMSIPreambleData
\newtheorem{theorem}{Theorem}
\newtheorem{acknowledgement}[theorem]{Acknowledgement}
\newtheorem{algorithm}[theorem]{Algorithm}
\newtheorem{axiom}[theorem]{Axiom}
\newtheorem{claim}[theorem]{Claim}
\newtheorem{conclusion}[theorem]{Conclusion}
\newtheorem{condition}[theorem]{Condition}
\newtheorem{conjecture}[theorem]{Conjecture}
\newtheorem{corollary}[theorem]{Corollary}
\newtheorem{criterion}[theorem]{Criterion}
\newtheorem{definition}[theorem]{Definition}
\newtheorem{example}[theorem]{Example}
\newtheorem{exercise}[theorem]{Exercise}
\newtheorem{lemma}[theorem]{Lemma}
\newtheorem{notation}[theorem]{Notation}
\newtheorem{problem}[theorem]{Problem}
\newtheorem{proposition}[theorem]{Proposition}
\newtheorem{remark}[theorem]{Remark}
\newtheorem{solution}[theorem]{Solution}
\newtheorem{summary}[theorem]{Summary}
\newenvironment{proof}[1][Proof]{\noindent\textbf{#1.} }{\ \rule{0.5em}{0.5em}}
\begin{document}
\preprint{HEP/123-qed}
\title[Short title for running header]{REV\TeX 4-1}
\author{First Author}
\affiliation{My Institution}
\author{Second Author}
\affiliation{My Institution}
\author{Third Author}
\affiliation{Other Institution}
\keywords{one two three}
\pacs{PACS number}

\begin{abstract}
Shell document for REV\TeX\ 4-1.

\end{abstract}
\volumeyear{year}
\volumenumber{number}
\issuenumber{number}
\eid{identifier}
\date[Date text]{date}
\received[Received text]{date}

\revised[Revised text]{date}

\accepted[Accepted text]{date}

\published[Published text]{date}

\startpage{101}
\endpage{102}
\maketitle


\section{Lattice}

The lattice class encodes geometric information and provides geometric transformations.

\subsection{Data Members}

$%
\begin{array}
[c]{ccc}%
\text{long} & \text{\textbf{Nx\_,Ny\_,Nz\_}} & \text{number of points in each
direction}\\
\text{long} & \text{\textbf{Ntot\_,Nout\_}} & \text{total number of points in
real and in Fourier space*}\\
\text{double} & \text{\textbf{dx\_,dy\_,dz\_}} & \text{spacing in each
direction}\\
\text{double} & \text{\textbf{L\_[3]}} & \text{dimension of total cell in each
direction}%
\end{array}
$

*In Fourier space, there are (about) half as many points as in real space
since each entry is complex and so has two degrees of freedom. This is
calculated as $\ N_{x}\times N_{y}\times\left(  \left[  \frac{N_{z}}%
{2}\right]  +1\right)  $ in accord with FFTW. 

\subsection{Accessors}

$%
\begin{array}
[c]{ccc}%
\text{double} & \text{\textbf{getDX()},...} & \text{accessors for dx\_,...}\\
\text{long} & \text{\textbf{Nx()},...} & \text{accessors for Nx\_,...}\\
\text{double} & \text{\textbf{Lx()},...} & \text{accessors for L\_[0],...}\\
\text{long} & \text{\textbf{Ntot()},\textbf{Nout}()} & \text{accessors for
Ntot\_,Nout\_}\\
\text{long} & \text{\textbf{size()}} & \text{same as \textbf{Ntot()}}\\
\text{double} & \text{\textbf{dV}(),\textbf{getVolume}()} & dx\_dy\_dz\_\text{
and }L\_[0]L\_[1]L\_[2]\\
\text{double} & \text{\textbf{getX(}int pos\textbf{)},...} & dx\_\times\left(
pos-\frac{Nx\_-1}{2}\right)  \text{,...}%
\end{array}
$

\subsection{Conversions}

$%
\begin{array}
[c]{cc}%
\text{long \textbf{pos(}int v[]\textbf{)},\textbf{pos(}int ix, int iy, int
iz)} & \text{convert cartesian indices into serial index}\\
\text{void \textbf{cartesian(}long pos, int ix, int iy, int iz\textbf{)}} &
\text{convert serial index to cartesian indices}\\%
\begin{array}
[c]{c}%
\text{long \textbf{get\_PBC\_Pos(}int v[]\textbf{)}}\\
\text{long \textbf{get\_PBC\_Pos(}int ix, int iy, int iz\textbf{)}}%
\end{array}
& \text{cartesian--%
%TCIMACRO{\TEXTsymbol{>}}%
%BeginExpansion
$>$%
%EndExpansion
serial after applying PBCs }\\
\text{void \textbf{putIntoBox}(int \&ix, int \&iy, int\ \&iz)} & \text{apply
PBC to give }0\leq ix<N_{x}\text{,...}\\
\text{long \textbf{size()}} & \text{same as \textbf{Ntot()}}\\
\text{double \textbf{dV}(),\textbf{getVolume}()} & dx\_dy\_dz\_\text{ and
}L\_[0]L\_[1]L\_[2]\\
\text{double \textbf{getX(}int pos\textbf{)},...} & dx\_\times\left(
pos-\frac{Nx\_-1}{2}\right)  \text{,...}%
\end{array}
$


\end{document}