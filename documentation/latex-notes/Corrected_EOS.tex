\documentclass[aps,preprint]{revtex4-1}%
\usepackage{amssymb}
\usepackage{amsfonts}
\usepackage{amsmath}
\usepackage{graphicx}%
\setcounter{MaxMatrixCols}{30}
%TCIDATA{OutputFilter=latex2.dll}
%TCIDATA{Version=5.50.0.2960}
%TCIDATA{CSTFile=revtex4-1.cst}
%TCIDATA{Created=Friday, November 01, 2019 09:56:59}
%TCIDATA{LastRevised=Wednesday, November 27, 2019 12:32:59}
%TCIDATA{<META NAME="GraphicsSave" CONTENT="32">}
%TCIDATA{<META NAME="SaveForMode" CONTENT="1">}
%TCIDATA{BibliographyScheme=Manual}
%TCIDATA{<META NAME="DocumentShell" CONTENT="Articles\SW\REVTeX 4-1">}
%TCIDATA{ComputeDefs=
%$T=1.18$
%$\rho=0.317$
%$d=1.0$
%$\eta=\frac{\pi}{6}\rho d^{3}$
%$a=\frac{T}{\rho}\frac{1}{\left(  1-\eta\right)  ^{5}}\left(  -1+\eta
%+10\eta^{2}+6\eta^{3}-5\eta^{4}+\eta^{5}\right)  $
%$b=\frac{T}{6\rho^{2}}\frac{1}{\left(  1-\eta\right)  ^{5}}\left(
%1-5\eta-20\eta^{2}-4\eta^{3}+5\eta^{4}-\eta^{5}\right)  $
%}
%BeginMSIPreambleData
\providecommand{\U}[1]{\protect\rule{.1in}{.1in}}
%EndMSIPreambleData
\newtheorem{theorem}{Theorem}
\newtheorem{acknowledgement}[theorem]{Acknowledgement}
\newtheorem{algorithm}[theorem]{Algorithm}
\newtheorem{axiom}[theorem]{Axiom}
\newtheorem{claim}[theorem]{Claim}
\newtheorem{conclusion}[theorem]{Conclusion}
\newtheorem{condition}[theorem]{Condition}
\newtheorem{conjecture}[theorem]{Conjecture}
\newtheorem{corollary}[theorem]{Corollary}
\newtheorem{criterion}[theorem]{Criterion}
\newtheorem{definition}[theorem]{Definition}
\newtheorem{example}[theorem]{Example}
\newtheorem{exercise}[theorem]{Exercise}
\newtheorem{lemma}[theorem]{Lemma}
\newtheorem{notation}[theorem]{Notation}
\newtheorem{problem}[theorem]{Problem}
\newtheorem{proposition}[theorem]{Proposition}
\newtheorem{remark}[theorem]{Remark}
\newtheorem{solution}[theorem]{Solution}
\newtheorem{summary}[theorem]{Summary}
\newenvironment{proof}[1][Proof]{\noindent\textbf{#1.} }{\ \rule{0.5em}{0.5em}}
\begin{document}
\preprint{HEP/123-qed}
\title[Short title for running header]{REV\TeX 4-1}
\author{First Author}
\affiliation{My Institution}
\author{Second Author}
\affiliation{My Institution}
\author{Third Author}
\affiliation{Other Institution}
\keywords{one two three}
\pacs{PACS number}

\begin{abstract}
Shell document for REV\TeX\ 4-1.

\end{abstract}
\volumeyear{year}
\volumenumber{number}
\issuenumber{number}
\eid{identifier}
\date[Date text]{date}
\received[Received text]{date}

\revised[Revised text]{date}

\accepted[Accepted text]{date}

\published[Published text]{date}

\startpage{101}
\endpage{102}
\maketitle


The standard DFT model, or van der Waals model, for a given pair potential
$v\left(  r\right)  $ is
\begin{equation}
F\left[  \rho\right]  =F^{\text{HS}}\left(  d\left[  v\right]  ;\left[
\rho\right]  \right)  +\frac{1}{2}\int\rho\left(  \mathbf{r}_{1}\right)
\rho\left(  \mathbf{r}_{2}\right)  w_{\text{att}}\left(  r_{12}\right)
d\mathbf{r}_{1}d\mathbf{r}_{2}%
\end{equation}
where the potential is separted into a short-ranged repulsive part,
$v_{0}\left(  r\right)  $ , and a long-ranged attrative part, $w_{\text{att}%
}\left(  r_{12}\right)  $ where the former is used to calculate an effective
hard-sphere radius $d$ . There are two problems with this model. First, we
know the exact form of the functional for small densities, $\rho\left(
\mathbf{r}_{1}\right)  \rightarrow0$ , and this does not give the correct
limit. Second, it does not give a particularly good equation of state in the
uniform limit. It is therefore interesting to try to correct these problems.
Sometimes, I will assume that an accurate equation of state is known for the
uniform system:\ i.e., that we know lim$_{\rho\left(  \mathbf{r}\right)
\rightarrow\overline{\rho}}F\left[  \rho\right]  \equiv F\left(
\overline{\rho}\right)  =f\left(  \overline{\rho}\right)  V$ .

First, one could imagine adding a correction based on the FMT densities. The
vDW model gives an eos for the uniform fluid of
\begin{equation}
f^{\text{VDW}}\left(  \overline{\rho}\right)  =f^{\text{HS}}\left(
\overline{\rho}\right)  +\frac{1}{2}a_{\text{VDW}}\rho^{2}%
\end{equation}
The most natural to use is the local packing fraction, so if we have a known
EOS one could use, e.g.,%
\begin{align}
F\left[  \rho\right]    & =F^{\text{HS}}\left(  d\left[  v\right]  ;\left[
\rho\right]  \right)  +\frac{1}{2}\int\rho\left(  \mathbf{r}_{1}\right)
\rho\left(  \mathbf{r}_{2}\right)  w_{\text{att}}\left(  r_{12}\right)
d\mathbf{r}_{1}d\mathbf{r}_{2}\\
& +\int\left\{  f\left(  \frac{6}{\pi d^{3}}\eta\left(  \mathbf{r}\right)
\right)  -f^{\text{VDW}}\left(  \frac{6}{\pi d^{3}}\eta\left(  \mathbf{r}%
\right)  \right)  \right\}  d\mathbf{r}\nonumber
\end{align}
The problem with this is that in the solid we must evaluate these expressions
for $\eta\left(  \mathbf{r}\right)  $ very close to one, where the hs part
diverges. By subtracting out the vDW\ function at this packing fraction, we
risk screwing up the necessary divergence of  $F^{\text{HS}}$ and this is
unacceptable. We could use a different density - well, only $s\left(
\mathbf{r}\right)  $ makes any sense - but then the problem is that this will
give essentially zero for $\mathbf{r}$ at a lattice site and will have a
large-ish value for positions have a hard-sphere diamter from a lattice
position. Perhaps this is OK but it sounds strange and perhaps dangerous. It
may be for these reasons that at least one group has used a correction of this
form based on an entirely new density,
\[
\widetilde{\rho}\left(  \mathbf{r}_{1}\right)  =\frac{3}{4\pi d^{3}}\int%
\Theta\left(  d-r_{12}\right)  \rho\left(  \mathbf{r}_{2}\right)  d\mathbf{r}%
\]
which is the same as $\eta\left(  \mathbf{r}\right)  $ but averaged over a
larger volume (the latter is averaged over a radius of d/2). I find this
unobjectionable from a technical point of view. It suffers however from the
fact that you are going to need the liquid eos at high densities, including at
low temperatures where the liquid does not exist. An alternative is to write a
simpler correction, e.g.
\begin{equation}
F\left[  \rho\right]  =F^{\text{HS}}\left(  d\left[  v\right]  ;\left[
\rho\right]  \right)  +\frac{1}{2}\int\rho\left(  \mathbf{r}_{1}\right)
\rho\left(  \mathbf{r}_{2}\right)  w_{\text{att}}\left(  r_{12}\right)
d\mathbf{r}_{1}d\mathbf{r}_{2}+b\int\widetilde{\rho}\left(  \mathbf{r}%
_{1}\right)  ^{3}d\mathbf{r}_{1}%
\end{equation}
and in fact, I would advocate going one step further with
\begin{equation}
F\left[  \rho\right]  =F^{\text{HS}}\left(  d\left[  v\right]  ;\left[
\rho\right]  \right)  +\frac{1}{2}\frac{a}{a_{\text{vdw}}}\int\rho\left(
\mathbf{r}_{1}\right)  \rho\left(  \mathbf{r}_{2}\right)  w_{\text{att}%
}\left(  r_{12}\right)  d\mathbf{r}_{1}d\mathbf{r}_{2}+b\int\widetilde{\rho
}\left(  \mathbf{r}_{1}\right)  ^{3}d\mathbf{r}_{1}%
\end{equation}
giving the eos for the liquid%
\begin{equation}
\beta f\left(  \overline{\rho}\right)  \equiv\beta\frac{1}{V}F\left(
\overline{\rho}\right)  =\overline{\rho}\ln\overline{\rho}\Lambda
^{3}-\overline{\rho}+\frac{\overline{\eta}\left(  4-3\overline{\eta}\right)
}{\left(  1-\overline{\eta}\right)  ^{2}}\overline{\rho}+\beta a\overline
{\rho}^{2}+\beta b\overline{\rho}^{3}%
\end{equation}
Now, one very simple way to fix the exponents is by matching the critical
point. For example, from this we know that the chemical potential and pressure
are%
\begin{align*}
\beta\mu & =\frac{\partial\beta f\left(  \overline{\rho}\right)  }%
{\partial\overline{\rho}}=\ln\overline{\rho}\Lambda^{3}+\frac{\overline{\eta
}\left(  8-9\overline{\eta}+3\overline{\eta}^{2}\right)  }{\left(
1-\overline{\eta}\right)  ^{3}}+2\beta a\overline{\rho}+3\beta b\overline
{\rho}^{2}\\
\beta P  & =-\beta f+\beta\mu\overline{\rho}=\frac{1+\overline{\eta}%
+\overline{\eta}^{2}-\overline{\eta}^{3}}{\left(  1-\overline{\eta}\right)
^{3}}\overline{\rho}+\beta a\overline{\rho}^{2}+2\beta b\overline{\rho}^{3}%
\end{align*}
and now we use that the critical point is an inflection point of the pressure:%
\begin{align}
0  & =\frac{\partial}{\partial\overline{\rho}}P=\frac{1+4\overline{\eta}%
_{c}+4\overline{\eta}_{c}^{2}-4\overline{\eta}_{c}^{3}+\overline{\eta}_{c}%
^{4}}{\left(  1-\overline{\eta}_{c}\right)  ^{4}}+2\beta_{c}a\overline{\rho
}_{c}+6\beta_{c}b\overline{\rho}_{c}^{2}\nonumber\\
0  & =\frac{\partial^{2}}{\partial\overline{\rho}^{2}}P=\frac{\pi d^{3}}%
{6}\frac{4\left(  2+5\overline{\eta}_{c}-\overline{\eta}_{c}^{2}\right)
}{\left(  1-\overline{\eta}_{c}\right)  ^{5}}+2\beta_{c}a+12\beta
_{c}b\overline{\rho}_{c}%
\end{align}
So if we know the critical temperature and density, we can simply solve to get%
\begin{align*}
b  & =\frac{k_{B}T_{c}}{6\overline{\rho}_{c}^{2}}\frac{1}{\left(  1-\eta
_{c}\right)  ^{5}}\left(  1-5\eta_{c}-20\eta_{c}^{2}-4\eta_{c}^{3}+5\eta
_{c}^{4}-\eta_{c}^{5}\right)  \allowbreak\\
a  & =\frac{k_{B}T_{c}}{\overline{\rho}_{c}}\frac{1}{\left(  1-\eta
_{c}\right)  ^{5}}\left(  -1+\eta_{c}+10\eta_{c}^{2}+6\eta_{c}^{3}-5\eta
_{c}^{4}+\eta_{c}^{5}\right)  \allowbreak
\end{align*}


\bigskip%
\begin{align*}
T  & =1.18\\
\rho & =0.317\\
d  & =1.0\\
\eta & =\frac{\pi}{6}\rho d^{3}\\
a  & =\frac{T}{\rho}\frac{1}{\left(  1-\eta\right)  ^{5}}\left(
-1+\eta+10\eta^{2}+6\eta^{3}-5\eta^{4}+\eta^{5}\right)  \\
b  & =\frac{T}{6\rho^{2}}\frac{1}{\left(  1-\eta\right)  ^{5}}\left(
1-5\eta-20\eta^{2}-4\eta^{3}+5\eta^{4}-\eta^{5}\right)
\end{align*}
%

\begin{align*}
a\beta & =-4.\,\allowbreak18\\
b\beta & =-1.\,\allowbreak626
\end{align*}


\bigskip

\bigskip

\bigskip

Assuming standrd FMT treatment of the hard-sphere part, this gives the
uniform-system eos%
\begin{equation}
\frac{1}{V}\beta F_{\text{ex}}\left(  \rho\right)  =\rho\frac{\eta\left(
4-3\eta\right)  }{\left(  1-\eta\right)  ^{2}}+\frac{1}{2}a\left[  v\right]
\rho^{2}%
\end{equation}
and the low-density functional%
\begin{equation}
\frac{1}{V}\beta F_{\text{ex}}\left[  \rho\right]  =\frac{1}{2V}\int%
\rho\left(  \mathbf{r}_{1}\right)  \rho\left(  \mathbf{r}_{2}\right)  \left\{
\Theta\left(  d\left[  v\right]  -r_{12}\right)  +\beta w_{\text{att}}\left(
r_{12}\right)  \right\}  d\mathbf{r}_{1}d\mathbf{r}_{2}+O\left[  \rho
^{3}\right]
\end{equation}
Now, we know that the exact limit is
\begin{equation}
\frac{1}{V}\beta F_{\text{ex}}\left[  \rho\right]  =\frac{1}{2V}\int%
\rho\left(  \mathbf{r}_{1}\right)  \rho\left(  \mathbf{r}_{2}\right)  f\left(
r_{12};\left[  v\right]  \right)  d\mathbf{r}_{1}d\mathbf{r}_{2}+O\left[
\rho^{3}\right]
\end{equation}
where%
\begin{equation}
f\left(  r;\left[  v\right]  \right)  =1-e^{-\beta v\left(  r\right)  }.
\end{equation}
This could be recovered by replacing
\begin{equation}
\beta w_{\text{att}}\left(  r_{12}\right)  \rightarrow\beta w_{\text{eff}%
}\left(  r_{12}\right)  \equiv f\left(  r_{12};\left[  v\right]  \right)
-\Theta\left(  d\left[  v\right]  -r_{12}\right)
\end{equation}
giving what I will call Model A. It clearly has the virtue of achieving the
zero-density limit in the most natural way. However, this forces
$w_{\text{eff}}$ to be zero inside the core (i.e. for $r$ near zero) which is
quite different from $\beta w_{\text{att}}$ and could lead to problems.
Alternatively, we could introduce this correction more heuristically as%
\begin{align}
F_{\text{ex}}\left[  \rho\right]   &  =F_{\text{ex}}^{\text{HS}}\left(
d\left[  v\right]  ;\left[  \rho\right]  \right)  +\frac{1}{2}\int\rho\left(
\mathbf{r}_{1}\right)  \rho\left(  \mathbf{r}_{2}\right)  w_{\text{att}%
}\left(  r_{12}\right)  d\mathbf{r}_{1}d\mathbf{r}_{2}\\
&  +A\left(  \frac{1}{2V}\int\rho\left(  \mathbf{r}_{1}\right)  \rho\left(
\mathbf{r}_{2}\right)  \beta w_{\text{att}}\left(  r_{12}\right)
d\mathbf{r}_{1}d\mathbf{r}_{2}\right) \nonumber\\
&  \times\left\{  \frac{1}{2V}\int\rho\left(  \mathbf{r}_{1}\right)
\rho\left(  \mathbf{r}_{2}\right)  \left\{  f\left(  r_{12};\left[  v\right]
\right)  -\Theta\left(  d\left[  v\right]  -r_{12}\right)  -\beta
w_{\text{att}}\left(  r_{12}\right)  \right\}  d\mathbf{r}_{1}d\mathbf{r}%
_{2}\right\} \nonumber
\end{align}
The idea is that we do not touch the hard sphere part, we allow the vdW part
to at least have the possiblity of recovering the "usual" form at high
densities (simply by haveing $A$ go to zero) and the zero-density limit just
says that $A(0)=1$ . This is Model B. Finally, we could forget the explicit
form of the zero-density dcf altogether and simply write
\begin{equation}
F_{\text{ex}}\left[  \rho\right]  =F_{\text{ex}}^{\text{HS}}\left(  d\left[
v\right]  ;\left[  \rho\right]  \right)  +A\left(  \frac{1}{2V}\int\rho\left(
\mathbf{r}_{1}\right)  \rho\left(  \mathbf{r}_{2}\right)  \beta w_{\text{att}%
}\left(  r_{12}\right)  d\mathbf{r}_{1}d\mathbf{r}_{2}\right)
\end{equation}
which I call Model C. The advantage here is that this is simpler than Model B,
can still go over to the vdW model and the mean field term is completely
independent of the hard sphere stuff.

The uniform limit of these three models are
\begin{align}
\frac{1}{V}\beta F_{\text{ex}}^{\left(  \text{A}\right)  }\left(  \rho\right)
&  =\rho\frac{\eta^{2}\left(  5-4\eta\right)  }{\left(  1-\eta\right)  ^{2}%
}+\rho^{2}B_{2}\left[  v\right] \\
\frac{1}{V}\beta F_{\text{ex}}^{\left(  \text{B}\right)  }\left(  \rho\right)
&  =\rho\frac{\eta\left(  4-3\eta\right)  }{\left(  1-\eta\right)  ^{2}}%
+\frac{1}{2}\rho^{2}a^{\left(  \text{vdW}\right)  }+A\left(  \frac{1}{2}%
\rho^{2}a^{\left(  \text{vdW}\right)  }\right)  \left\{  -4\eta\rho-\frac
{1}{2}\rho^{2}a^{\left(  \text{vdW}\right)  }+\rho^{2}B_{2}\right\}
\nonumber\\
\frac{1}{V}\beta F_{\text{ex}}^{\left(  \text{C}\right)  }\left(  \rho\right)
&  =\rho\frac{\eta\left(  4-3\eta\right)  }{\left(  1-\eta\right)  ^{2}%
}+A\left(  \frac{1}{2}\rho^{2}a^{\left(  \text{vdW}\right)  }\right) \nonumber
\end{align}
with%
\begin{align}
a^{\left(  \text{vdW}\right)  }  &  =\int\beta w_{\text{att}}\left(  r\right)
d\mathbf{r}\\
B_{2}\left[  v\right]   &  =\frac{1}{2}\int f\left(  r_{12};\left[  v\right]
\right)  d\mathbf{r}\nonumber
\end{align}
To summarize%
\begin{align}
\frac{1}{V}\beta F_{\text{ex}}^{\left(  \text{vdW}\right)  }\left[
\rho\right]   &  =F^{\text{HS}}\left(  d\left[  v\right]  ;\left[
\rho\right]  \right)  +\frac{1}{2}\int\rho\left(  \mathbf{r}_{1}\right)
\rho\left(  \mathbf{r}_{2}\right)  w_{\text{att}}\left(  r_{12}\right)
d\mathbf{r}_{1}d\mathbf{r}_{2}\\
\frac{1}{V}\beta F_{\text{ex}}^{\left(  \text{A}\right)  }\left[  \rho\right]
&  =F^{\text{HS}}\left(  d\left[  v\right]  ;\left[  \rho\right]  \right)
+\frac{1}{2}\int\rho\left(  \mathbf{r}_{1}\right)  \rho\left(  \mathbf{r}%
_{2}\right)  \left\{  f\left(  r_{12};\left[  v\right]  \right)
-\Theta\left(  d\left[  v\right]  -r_{12}\right)  \right\}  d\mathbf{r}%
_{1}d\mathbf{r}_{2}\nonumber\\
\frac{1}{V}\beta F_{\text{ex}}^{\left(  \text{B}\right)  }\left[  \rho\right]
&  =F^{\text{HS}}\left(  d\left[  v\right]  ;\left[  \rho\right]  \right)
+\frac{1}{2}\int\rho\left(  \mathbf{r}_{1}\right)  \rho\left(  \mathbf{r}%
_{2}\right)  w_{\text{att}}\left(  r_{12}\right)  d\mathbf{r}_{1}%
d\mathbf{r}_{2}\nonumber\\
&  +A\left[  \rho\right]  \left\{  \frac{1}{2V}\int\rho\left(  \mathbf{r}%
_{1}\right)  \rho\left(  \mathbf{r}_{2}\right)  \left\{  f\left(
r_{12};\left[  v\right]  \right)  -\Theta\left(  d\left[  v\right]
-r_{12}\right)  -\beta w_{\text{att}}\left(  r_{12}\right)  \right\}
d\mathbf{r}_{1}d\mathbf{r}_{2}\right\} \nonumber\\
\frac{1}{V}\beta F_{\text{ex}}^{\left(  \text{C}\right)  }\left[  \rho\right]
&  =F^{\text{HS}}\left(  d\left[  v\right]  ;\left[  \rho\right]  \right)
+A\left[  \rho\right] \nonumber
\end{align}
with%
\begin{equation}
A\left[  \rho\right]  =A\left(  \frac{1}{2V}\int\rho\left(  \mathbf{r}%
_{1}\right)  \rho\left(  \mathbf{r}_{2}\right)  \beta w_{\text{att}}\left(
r_{12}\right)  d\mathbf{r}_{1}d\mathbf{r}_{2}\right)
\end{equation}
One last possibility suggests itself: we could get rid of the attractive
potential altogether and simply write%
\begin{equation}
\frac{1}{V}\beta F_{\text{ex}}^{\left(  \text{D}\right)  }\left[  \rho\right]
=F^{\text{HS}}\left(  d\left[  v\right]  ;\left[  \rho\right]  \right)
+B\left[  \rho\right]
\end{equation}
with%
\begin{equation}
B^{\left(  \text{D1}\right)  }\left[  \rho\right]  =B\left(  \frac{1}{2V}%
\int\rho\left(  \mathbf{r}_{1}\right)  \rho\left(  \mathbf{r}_{2}\right)
\left\{  f\left(  r_{12};\left[  v\right]  \right)  -\Theta\left(  d\left[
v\right]  -r_{12}\right)  \right\}  d\mathbf{r}_{1}d\mathbf{r}_{2}\right)
\end{equation}
to get the zero-density limit correctly or, more simply,
\begin{equation}
B^{\left(  \text{D2}\right)  }\left[  \rho\right]  =B\left(  \frac{1}{2V}%
\int\rho\left(  \mathbf{r}_{1}\right)  \rho\left(  \mathbf{r}_{2}\right)
f\left(  r_{12};\left[  v\right]  \right)  d\mathbf{r}_{1}d\mathbf{r}%
_{2}\right)
\end{equation}
with uniform limits%
\begin{equation}
\frac{1}{V}\beta F_{\text{ex}}^{\left(  \text{D1}\right)  }\left(
\rho\right)  =\rho\frac{\eta\left(  4-3\eta\right)  }{\left(  1-\eta\right)
^{2}}+B\left(  B_{2}\rho^{2}-4\eta\rho\right)
\end{equation}
and%
\begin{equation}
\frac{1}{V}\beta F_{\text{ex}}^{\left(  \text{D2}\right)  }\left(
\rho\right)  =\rho\frac{\eta\left(  4-3\eta\right)  }{\left(  1-\eta\right)
^{2}}+B\left(  B_{2}\rho^{2}\right)
\end{equation}
respectively.

So, given all these possibilities, what is the best?\ First, despite the
different lengths of the expressions, all of these involve pretty much the
same computational (and coding) effort: just the typical two-body terms. Model
B is a little worse since we would have to compute, effectively, two
potentials. Otherwise, it seems to me that model B, while somewhat
complicated, is the most conservative since it gives the correct low-density
limit while not changing too much the high-density behaviour from the standard
vDW model. On the other hand, model D1 seems conceptually the nicest - it
gives the correct low-density limit while introducing the fewest ad hoc
elements. Finally, of all of these proposals, model C is the minimal change
from the vDW model.
\end{document}