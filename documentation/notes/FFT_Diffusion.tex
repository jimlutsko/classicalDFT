\documentclass[aps,preprint]{revtex4-1}%
\usepackage{amssymb}
\usepackage{amsfonts}
\usepackage{amsmath}
\usepackage{graphicx}%
\setcounter{MaxMatrixCols}{30}
%TCIDATA{OutputFilter=latex2.dll}
%TCIDATA{Version=5.50.0.2960}
%TCIDATA{CSTFile=revtex4-1.cst}
%TCIDATA{Created=Friday, June 15, 2018 09:27:26}
%TCIDATA{LastRevised=Tuesday, June 26, 2018 14:26:32}
%TCIDATA{<META NAME="GraphicsSave" CONTENT="32">}
%TCIDATA{<META NAME="SaveForMode" CONTENT="1">}
%TCIDATA{BibliographyScheme=Manual}
%TCIDATA{<META NAME="DocumentShell" CONTENT="Articles\SW\REVTeX 4-1">}
%BeginMSIPreambleData
\providecommand{\U}[1]{\protect\rule{.1in}{.1in}}
%EndMSIPreambleData
\newtheorem{theorem}{Theorem}
\newtheorem{acknowledgement}[theorem]{Acknowledgement}
\newtheorem{algorithm}[theorem]{Algorithm}
\newtheorem{axiom}[theorem]{Axiom}
\newtheorem{claim}[theorem]{Claim}
\newtheorem{conclusion}[theorem]{Conclusion}
\newtheorem{condition}[theorem]{Condition}
\newtheorem{conjecture}[theorem]{Conjecture}
\newtheorem{corollary}[theorem]{Corollary}
\newtheorem{criterion}[theorem]{Criterion}
\newtheorem{definition}[theorem]{Definition}
\newtheorem{example}[theorem]{Example}
\newtheorem{exercise}[theorem]{Exercise}
\newtheorem{lemma}[theorem]{Lemma}
\newtheorem{notation}[theorem]{Notation}
\newtheorem{problem}[theorem]{Problem}
\newtheorem{proposition}[theorem]{Proposition}
\newtheorem{remark}[theorem]{Remark}
\newtheorem{solution}[theorem]{Solution}
\newtheorem{summary}[theorem]{Summary}
\newenvironment{proof}[1][Proof]{\noindent\textbf{#1.} }{\ \rule{0.5em}{0.5em}}
\begin{document}
\preprint{HEP/123-qed}
\title[Short title for running header]{REV\TeX 4-1}
\author{First Author}
\affiliation{My Institution}
\author{Second Author}
\affiliation{My Institution}
\author{Third Author}
\affiliation{Other Institution}
\keywords{one two three}
\pacs{PACS number}

\begin{abstract}
Shell document for REV\TeX\ 4-1.

\end{abstract}
\volumeyear{year}
\volumenumber{number}
\issuenumber{number}
\eid{identifier}
\date[Date text]{date}
\received[Received text]{date}

\revised[Revised text]{date}

\accepted[Accepted text]{date}

\published[Published text]{date}

\startpage{101}
\endpage{102}
\maketitle


\section{Diffusion}

Let the discretized density be%
\begin{equation}
\rho_{lmn}^{t}=\rho\left(  \mathbf{r}_{lmn},t\right)  ,\;\mathbf{r}%
_{lmn}=l\Delta\widehat{\mathbf{x}}+m\Delta\widehat{\mathbf{x}}+n\Delta
\widehat{\mathbf{x}}%
\end{equation}
where, for simplicity, I take
\begin{equation}
0\leq l,m,n\leq N-1.
\end{equation}
Consider a discretized diffusion equation with periodic boundaries:%
\begin{equation}
\partial_{t}\rho_{lmn}^{t}=D\left(  \frac{\rho_{l+1mn}^{t}+\rho_{l-1mn}%
^{t}-2\rho_{lmn}^{t}}{\Delta^{2}}+...\right)
\end{equation}
Now form the DFT%
\begin{equation}
\widetilde{\rho}_{abc}^{t}=\sum_{lmn}\rho_{lmn}^{t}e^{i\left(
al+bm+cn\right)  2\pi/N}%
\end{equation}
Then, using%
\begin{align*}
\sum_{l=0}^{N-1}e^{ial2\pi/N}\rho_{l+w}^{t}  & =\sum_{l=w}^{N-1+w}e^{ia\left(
l-w\right)  2\pi/N}\rho_{l}^{t}\\
& =e^{-iaw2\pi/N}\sum_{l=w}^{N-1+w}e^{ial2\pi/N}\rho_{l}^{t}\\
& =e^{-iaw2\pi/N}\left(  \sum_{l=w}^{N-1}e^{ial2\pi/N}\rho_{l}^{t}+\sum
_{l=N}^{N-1+w}e^{ial2\pi/N}\rho_{l}^{t}\right)  \\
& =e^{-iaw2\pi/N}\left(  \sum_{l=w}^{N-1}e^{ial2\pi/N}\rho_{l}^{t}+\sum
_{l=0}^{w-1}e^{ia\left(  l+N\right)  2\pi/N}\rho_{l+N}^{t}\right)  \\
& =e^{-iaw2\pi/N}\sum_{l=0}^{N-1}e^{ial2\pi/N}\rho_{l}^{t}%
\end{align*}
gives%
\begin{align}
\partial_{t}\widetilde{\rho}_{abc}^{t}  & =\frac{D}{\Delta^{2}}\left(  \left(
e^{-ia2\pi/N}+e^{ia2\pi/N}-2\right)  +\left(  e^{-ib2\pi/N}+e^{ib2\pi
/N}-2\right)  +\left(  e^{-ic2\pi/N}+e^{ic2\pi/N}-2\right)  \right)
\widetilde{\rho}_{abc}^{t}\\
& =\frac{2D}{\Delta^{2}}\left(  \left(  \cos\left(  \frac{2\pi a}{N}\right)
-1\right)  +\left(  \cos\left(  \frac{2\pi b}{N}\right)  -1\right)  +\left(
\cos\left(  \frac{2\pi c}{N}\right)  -1\right)  \right)  \widetilde{\rho
}_{abc}^{t}\nonumber\\
& =\frac{4D}{\Delta^{2}}\left(  \left(  \left(  \cos\left(  \frac{\pi a}%
{N}\right)  \right)  ^{2}-1\right)  +\left(  \left(  \cos\left(  \frac{\pi
b}{N}\right)  \right)  ^{2}-1\right)  +\left(  \left(  \cos\left(  \frac{\pi
c}{N}\right)  \right)  ^{2}-1\right)  \right)  \widetilde{\rho}_{abc}%
^{t}\nonumber\\
& \equiv\Lambda_{abc}\widetilde{\rho}_{abc}^{t}%
\end{align}
and so, evidently,
\begin{equation}
\widetilde{\rho}_{abc}^{t}=U_{abc}^{t}\widetilde{\rho}_{abc}^{0}%
\end{equation}
with%
\begin{equation}
U_{abc}^{t}=\exp\left(  \frac{4D}{\Delta^{2}}\Lambda_{abc}t\right)
\end{equation}


\bigskip an inverse DFT then gives the exact $\rho_{lmn}^{t}$. 

With a source term, we have that%
\begin{equation}
\partial_{t}\widetilde{\rho}_{abc}^{t}=\Lambda_{abc}\widetilde{\rho}_{abc}%
^{t}+\widetilde{F}_{abc}^{t}%
\end{equation}
so%
\begin{equation}
\partial_{t}e^{-\Lambda_{abc}t}\widetilde{\rho}_{abc}^{t}=e^{-\Lambda_{abc}%
t}\widetilde{F}_{abc}^{t}%
\end{equation}
or%
\begin{equation}
\widetilde{\rho}_{abc}^{t}=e^{\Lambda_{abc}t}\widetilde{\rho}_{abc}%
^{0}+e^{\Lambda_{abc}t}\int_{0}^{t}e^{-\Lambda_{abc}\tau}\widetilde{F}%
_{abc}^{\tau}d\tau
\end{equation}
Now, we get a semi-implicit scheme with%
\begin{equation}
\widetilde{F}_{abc}^{\tau}\simeq\widetilde{F}_{abc}^{0}+\frac{\tau}{t}\left(
\widetilde{F}_{abc}^{t}-\widetilde{F}_{abc}^{0}\right)
\end{equation}
giving%
\begin{equation}
\widetilde{\rho}_{abc}^{t}=e^{\Lambda_{abc}t}\widetilde{\rho}_{abc}^{0}%
+\frac{e^{\Lambda_{abc}t}-1}{\Lambda_{abc}}\widetilde{F}_{abc}^{0}%
+\frac{e^{\Lambda_{abc}t}-1-t\Lambda_{abc}}{\Lambda_{abc}^{2}t}\left(
\widetilde{F}_{abc}^{t}-\widetilde{F}_{abc}^{0}\right)
\end{equation}
and
\begin{equation}
\lim_{\Lambda_{abc}\rightarrow0}\widetilde{\rho}_{abc}^{t}=\widetilde{\rho
}_{abc}^{0}+\frac{1}{2}t\left(  \widetilde{F}_{abc}^{t}+\widetilde{F}%
_{abc}^{0}\right)
\end{equation}
which is the usual (C-N) semi-implicit scheme. Since the time-dependence of
the source actually comes from the density, we are left with%
\begin{equation}
\widetilde{\mathbf{\rho}}^{t}=e^{\mathbf{\Lambda}t}\widetilde{\mathbf{\rho}%
}^{0}+\frac{e^{\mathbf{\Lambda}t}-1}{\mathbf{\Lambda}}\widetilde{\mathbf{F}%
}\left[  \widetilde{\mathbf{\rho}}^{0}\right]  +\frac{e^{\mathbf{\Lambda}%
t}-1-t\mathbf{\Lambda}}{\mathbf{\Lambda}t}\left(  \widetilde{\mathbf{F}%
}\left[  \widetilde{\mathbf{\rho}}^{t}\right]  -\widetilde{\mathbf{F}}\left[
\widetilde{\mathbf{\rho}}^{0}\right]  \right)
\end{equation}
which should be solved, e.g., by Picard iteration. 

\section{Opening the system}

A cheap way to go from a closed system to an open one is to fix the density on
the boundaries. In this way, material can enter and leave the system with no
problem. If we continue to use periodic boundaries otherwise, any stationary
point of the closed system will also be a stationary point of the open system,
but it goes from being stable to unstable. To see the effect of this, let us
go back to the diffusion problem in one dimension%
\begin{align}
\partial_{t}\rho_{0}^{t}  & =0\\
\partial_{t}\rho_{l}^{t}  & =D\left(  \frac{\rho_{l+1}^{t}+\rho_{l-1}%
^{t}-2\rho_{l}^{t}}{\Delta^{2}}+...\right)  \nonumber
\end{align}
and now multiply each equation by $e^{ial2\pi/N}$ and sum to get%
\begin{align*}
\partial_{t}\widetilde{\rho}_{a}^{t}  & =\frac{D}{\Delta^{2}}\left(
\sum_{l=1}^{N-1}e^{ial2\pi/N}\rho_{l+1}^{t}+\sum_{l=1}^{N-1}e^{ial2\pi/N}%
\rho_{l+1}^{t}-2\sum_{l=1}^{N-1}e^{ial2\pi/N}\rho_{l}^{t}\right)  \\
& =\frac{D}{\Delta^{2}}\left(  \sum_{l=0}^{N-1}e^{ial2\pi/N}\rho_{l+1}%
^{t}+\sum_{l=0}^{N-1}e^{ial2\pi/N}\rho_{l+1}^{t}-2\sum_{l=0}^{N-1}%
e^{ial2\pi/N}\rho_{l}^{t}\right)  -\frac{D}{\Delta^{2}}\left(  \rho_{1}%
^{t}+\rho_{N-1}^{t}-2\rho_{0}^{t}\right)  \\
& =-\frac{4D}{\Delta^{2}}\left(  \sin\left(  \frac{\pi a}{N}\right)  \right)
^{2}\widetilde{\rho}_{a}^{t}-\frac{D}{\Delta^{2}}\left(  \rho_{1}^{t}%
+\rho_{N-1}^{t}-2\rho_{0}^{t}\right)
\end{align*}
which can no longer be solved as before since: all of the Fourier components
are coupled. In one dimension, this is not really a problem. If the BC is
$\rho_{0}^{t}=\rho_{0}$ and $\rho_{N}^{t}=\rho_{N}$ then we simply form
$\rho_{t}^{\prime}\left(  x\right)  =\rho_{t}\left(  x\right)  -\left(
\rho_{0}+\frac{\rho_{N}-\rho_{0}}{L}x\right)  $ and we find that $\rho
_{t}^{\prime}\left(  x\right)  $ satisfies the diffusion equation with
$\rho_{t}^{\prime}\left(  0\right)  =\rho_{t}^{\prime}\left(  L\right)  =0$
which just means that the FT is restricted to the sin transform. 

Here, we will try an alternative method. We will simply group the
inhomogeneous terms into the source term and continue to treat it as before.
Hence, in the 3-D case, we will have the same equations with
\[
F_{lmn}^{t}\rightarrow F_{lmn}^{t}-\frac{D}{\Delta^{2}}\left\{  \left(
\delta_{l1}+\delta_{lN-1}-2\delta_{l0}\right)  +\left(  \delta_{m1}%
+\delta_{mN-1}-2\delta_{m0}\right)  +\left(  \delta_{n1}+\delta_{nN-1}%
-2\delta_{n0}\right)  \right\}  \rho_{lmn}%
\]


Alternatively, we use a sine transform defining
\[
\widetilde{\rho}_{a}^{t}=\sum_{l=1}^{N-1}\sin\left(  \frac{al}{N}\pi\right)
e^{ial2\pi/N}\rho_{l}^{t}%
\]
so that
\begin{align*}
\partial_{t}\widetilde{\rho}_{a}^{t}  & =\frac{D}{\Delta^{2}}\left(
\sum_{l=1}^{N-1}\sin\left(  \frac{al}{N}\pi\right)  e^{ial2\pi/N}\rho
_{l+1}^{t}+\sum_{l=1}^{N-1}\sin\left(  \frac{al}{N}\pi\right)  e^{ial2\pi
/N}\rho_{l+1}^{t}-2\sum_{l=1}^{N-1}\sin\left(  \frac{al}{N}\pi\right)
e^{ial2\pi/N}\rho_{l}^{t}\right)  \\
& =\frac{D}{\Delta^{2}}\left(  \sum_{l=1}^{N-1}\sin\left(  \frac{al}{N}%
\pi\right)  e^{ial2\pi/N}\rho_{l+1}^{t}+\sum_{l=1}^{N-1}\sin\left(  \frac
{al}{N}\pi\right)  e^{ial2\pi/N}\rho_{l+1}^{t}-2\sum_{l=1}^{N-1}\sin\left(
\frac{al}{N}\pi\right)  e^{ial2\pi/N}\rho_{l}^{t}\right)
\end{align*}
Now%
\begin{align*}
\sum_{l=1}^{N-1}\sin\left(  \frac{al}{N}\pi\right)  \rho_{l+1}^{t}  &
=\sum_{l=2}^{N}\sin\left(  \frac{a\left(  l-1\right)  }{N}\pi\right)  \rho
_{l}^{t}\\
& =\sum_{l=1}^{N-1}\sin\left(  \frac{a\left(  l-1\right)  }{N}\pi\right)
\rho_{l}^{t}+\sin\left(  \frac{a\left(  N-1\right)  }{N}\pi\right)  \rho
_{N}^{t}%
\end{align*}
since the sin vanishes for $l=1$ so
\begin{align*}
\sum_{l=1}^{N-1}\sin\left(  \frac{al}{N}\pi\right)  \rho_{l+1}^{t}  &
=\cos\left(  \frac{a}{N}\pi\right)  \sum_{l=1}^{N-1}\sin\left(  \frac{al}%
{N}\pi\right)  \rho_{l}^{t}-\sin\left(  \frac{al}{N}\pi\right)  \sum
_{l=1}^{N-1}\cos\left(  \frac{al}{N}\pi\right)  \rho_{l}^{t}+\sin\left(
\frac{a\left(  N-1\right)  }{N}\pi\right)  \rho_{N}^{t}\\
\sum_{l=1}^{N-1}\sin\left(  \frac{al}{N}\pi\right)  \rho_{l-1}^{t}  &
=\sum_{l=0}^{N-2}\sin\left(  \frac{a\left(  l+1\right)  }{N}\pi\right)
\rho_{l}^{t}\\
& =\sum_{l=1}^{N-1}\sin\left(  \frac{a\left(  l+1\right)  }{N}\pi\right)
\rho_{l}^{t}+\sin\left(  \frac{1}{N}\pi\right)  \rho_{0}^{t}\\
& =\cos\left(  \frac{a}{N}\pi\right)  \sum_{l=1}^{N-1}\sin\left(  \frac{al}%
{N}\pi\right)  \rho_{l}^{t}+\sin\left(  \frac{a}{N}\pi\right)  \sum
_{l=1}^{N-1}\cos\left(  \frac{al}{N}\pi\right)  \rho_{l}^{t}+\sin\left(
\frac{a}{N}\pi\right)  \rho_{0}^{t}%
\end{align*}
and%
\begin{align*}
\sum_{l=1}^{N-1}\sin\left(  \frac{al}{N}\pi\right)  \rho_{l+1}^{t}+\sum
_{l=1}^{N-1}\sin\left(  \frac{al}{N}\pi\right)  \rho_{l-1}^{t}  &
=2\cos\left(  \frac{a}{N}\pi\right)  \sum_{l=1}^{N-1}\sin\left(  \frac{al}%
{N}\pi\right)  \rho_{l}^{t}+\sin\left(  \frac{a\left(  N-1\right)  }{N}%
\pi\right)  \rho_{N}^{t}+\sin\left(  \frac{a}{N}\pi\right)  \rho_{l}^{t}\\
& =2\cos\left(  \frac{a}{N}\pi\right)  \widetilde{\rho}_{a}^{t}+\sin\left(
a\pi\right)  \cos\left(  \frac{a}{N}\pi\right)  \rho_{N}^{t}-\sin\left(
\frac{a}{N}\pi\right)  \cos\left(  a\pi\right)  \rho_{N}^{t}+\sin\left(
\frac{a}{N}\pi\right)  \rho_{0}^{t}\\
& =2\cos\left(  \frac{a}{N}\pi\right)  \widetilde{\rho}_{a}^{t}+\sin\left(
\frac{a}{N}\pi\right)  \left(  \rho_{0}^{t}-\cos\left(  a\pi\right)  \rho
_{N}^{t}\right)
\end{align*}
and if $\rho_{0}^{t}=\rho_{N}^{t}=0$ this vanishes.

So for the open system, we must relax to get the background $\rho_{lmn}^{B}$as
well. Then we write
\[
u_{lmn}^{t}=\rho_{lmn}^{t}-\rho_{lmn}^{B}%
\]
and solve%
\begin{equation}
\partial_{t}\widetilde{u}_{abc}^{t}=\Lambda_{abc}^{\prime}\widetilde{u}%
_{abc}^{t}+\widetilde{F}_{abc}\left(  \mathbf{\rho}^{B}+\mathbf{u}^{t}\right)
\end{equation}
with%
\begin{align*}
\Lambda_{abc}^{\prime}  & =\frac{2D}{\Delta^{2}}\left(  \cos\left(  \frac
{a\pi}{N_{x}}\right)  -1\right)  \left(  \cos\left(  \frac{b\pi}{N_{y}%
}\right)  -1\right)  \left(  \cos\left(  \frac{c\pi}{N_{z}}\right)  -1\right)
\\
& =\frac{2D}{\Delta^{2}}\left\{  \left(  \cos\left(  \frac{a\pi}{N_{x}%
}\right)  -1\right)  +\left(  \cos\left(  \frac{b\pi}{N_{y}}\right)
-1\right)  +\left(  \cos\left(  \frac{c\pi}{N_{z}}\right)  -1\right)
\right\}  \\
& =-\frac{4D}{\Delta^{2}}\left\{  \left(  \sin\left(  \frac{a\pi}{2N_{x}%
}\right)  \right)  ^{2}+\left(  \sin\left(  \frac{b\pi}{2N_{y}}\right)
\right)  ^{2}+\left(  \sin\left(  \frac{c\pi}{2N_{z}}\right)  \right)
^{2}\right\}
\end{align*}



\end{document}